%=================================================================-*-LaTeX-*-==
% GRTensorIII 1.50 Manual
% Booklet E: Bases and tetrads
%
% Denis Pollney
% July 1996
%==============================================================================
\documentclass{article}
\usepackage{amsmath}
\usepackage{amssymb}
%\usepackage{maple2e}
\usepackage{longtable}
\usepackage{grtensor}
\usepackage{grbooks}
%------------------------------------------------------------------------------
\begin{document}
\grlabel{\grBasisLabel}
\grtitle{\grBasisTitle}
\grdate{Nov 2016}
\grtitlepage
%==============================================================================
% Body.
%==============================================================================
\copyrightpage
\noindent The classical method of describing a spacetime is through
the use of a covariant tensor, $g_{ab}$, which specifies the inner
product between a pair of vectors in the space. An alternative
description is obtained when one specifies a set of coordinate axes
(vectors) at each point of the spacetime, as well as their inner
product. Together, these are called a basis (or frame, or tetrad in
four dimensions).

For the most part, GRTensorIII calculations for a basis directly
parallel those for a metric and the two formalisms have been described
together wherever necessary in Booklets \textsl{A--D}. However, to
this point most of the examples used in these documents
have relied on calculations in metric coordinates. The goal
of this booklet is to supplement the previous descriptions by
emphasizing some of the unique features of symbolic calculations for
spacetimes described by a basis in GRTensorIII.
%
%------------------------------------------------------------------------------
\section{Notation}
%------------------------------------------------------------------------------
%
This section defines some notational conventions that are used to
describe bases in this set of booklets. A concise summary of basis
formalisms (using notation consistent with that described here) can be
found in the first part of Chandrasekhar \cite{chandrasekhar:1983}.\\

At each point of an $n$ dimensional space, define a set of $n$
independent vectors,
\[
  e_{(1)}{}^a, e_{(2)}{}^a, \ldots, e_{(n)}{}{}^a.  
\]
Following convention, the vectors are labeled using bracketed
indices, $(1), \ldots, (n)$, which we will call the `basis indices',
to distinguish them from the `tensor indices' which are not bracketed.
The basis indices are raised by considering $e_{(a)}{}^b$ to form an
$n\times n$ matrix whose inverse is given by the matrix $e^{(a)}{}_b$.

In addition to the basis vectors, define a symmetric $n\times n$
matrix $\eta^{(a)(b)}$ whose components determine the `inner
product' of basis vectors $(a)$ and $(b)$. The inner product and basis
vectors together can be used to define a 2-index tensor via
\begin{equation}
  \eta^{(a)(b)} e_{(a)}{}^c e_{(b)}{}^d =: g^{cd}.
  \label{eq:metric}
\end{equation}
This object, along with its matrix inverse, $g_{ab}$, is used to raise
and lower the tensor indices of the basis vectors:
\[
  e_{(a)b} := e_{(a)}{}^c g_{cb}, \qquad e_{(a)}{}^b = e_{(a)c} g^{cb}.
\]
Then by defining $\eta_{(a)(b)}$ to be the matrix inverse of
$\eta^{(a)(b)}$, some manipulation of the above relationships results
in the formulae for the raising and lowering of the basis indices:
\[
  e_{(a)b} = \eta_{(a)(c)} e^{(c)}{}_b, \qquad e^{(a)}{}^b =
  \eta^{(a)(c)} e_{(c)}{}^b
\] 

The collection of vectors, $e_{(a)}{}^b$, and inner product,
$\eta_{(a)(b)}$, together form a `basis' which locally describes the
spacetime geometry. The tensor $g_{ab}$ is the corresponding spacetime
metric that is calculated from the basis via Eq. (\ref{eq:metric}).\\

Tensors fields can be projected onto the basis field in order to
obtain their basis components:
\[
  T_{(a)(b)} = T_{cd} e_{(a)}{}^c e_{(b)}{}^d, \qquad
  T_{ab} = T_{(c)(d)} e^{(c)}{}_a e^{(d)}{}_b.
\]
The basis components of a tensor are raised and lowered
using the basis inner product,
\[
  T^{(a)}{}_{(b)} := \eta^{(a)(c)} T_{(c)(b)}, \qquad T^{(a)(b)} :=
  T^{(a)}{}_{(c)} \eta^{(c)(b)}.
\]
A computational advantage over tensor methods is offered here, since
by choosing basis vectors correctly the inner product $\eta_{(a)(b)}$
can be constructed so as to have a simple form, most often with
constant coefficients.\footnote{In GRTensorIII it is not necessary that
$\eta_{(a)(b)}$ have constant coefficients. See the demonstration
\texttt{mixmr.ms} available from the world-wide-web page
\cite{www}.}\\

The GRTensorIII notation for the basis vectors $e_{(a)}{}^b$ is
\texttt{e(bdn,up)}, following the convention that basis indices are
labeled using \texttt{bdn} and \texttt{bup} to indicate covariant and
contravariant components, respectively. The inner product,
$\eta_{(a)(b)}$, is referenced using the object name
\texttt{e(bdn,bdn)}. The creation of bases for GRTensorIII is outlined
in Booklet \grMakegRef.\\

A particularly useful form of basis (for general relativity) is given
by an inner product whose basis vectors are chosen to satisfy the inner
product
\[
  \eta^{(a)(b)} = \eta_{(a)(b)} :=
    \left[
      \begin{array}{cccc}
        0 & 1 & 0 & 0 \\
        1 & 0 & 0 & 0 \\
        0 & 0 & 0 &-1 \\
        0 & 0 &-1 & 0
      \end{array}
    \right]
\]
In this case the vectors forming the basis are seen to be null. By
convention, they are labeled
\[
  l^a := e_{(1)}{}^a, \qquad n^a := e_{(2)}{}^a, \qquad
  m^a := e_{(3)}{}^a, \qquad \bar{m}^a := e_{(4)}{}^a.
\]
A formalism for studying the geometries specified by such null tetrads
was developed by Newman and Penrose \cite{newman/penrose:1962}. As such,
they are often called `NP tetrads'. The standard GRTensorIII library
incorporates the full range of objects defined for the Newman-Penrose
formalism, as listed in Section \ref{sec:NP}.

Null tetrads can be created using \grcmd{makeg} or automatically from
a metric, $g_{ab}$, using the \grcmd{nptetrad} command. Rotations of
a null tetrad can be performed using the command \grcmd{nprotate}.
These commands are described in Booklet \grMakegRef.
%
%------------------------------------------------------------------------------
\section{Calculation of basis components}
%------------------------------------------------------------------------------
%
Calculations performed in a basis are carried out using the methods
described in Booklet \grCalcRef. A difference between the
calculation of basis components and tensor components of an object is
the use of basis index labels \texttt{bup} and \texttt{bdn} (to
indicate contravariant and covariant components, respectively) rather
than the tensor labels \texttt{up} and \texttt{dn}. For instance,
once a basis is loaded, the command\\

\noindent\texttt{> grcalc ( R(bdn,bdn) ):}\\

\noindent requests a calculation of the object $R_{(a)(b)}$, the
components of the Ricci tensor projected onto the basis.

Note that because the metric can be calculated from the basis via
Eq. (\ref{eq:metric}), if the geometry is described by a basis, then
either basis or tensor components of an object can be calculated. That
is, having loaded a basis one could also request the calculation\\

\noindent\texttt{> grcalc ( R(dn,dn) ):}\\

\noindent of the ordinary tensor components of the Ricci tensor, 
$R_{ab}$.\footnote{A word of caution should be noted when carrying out
such tensor component calculations from a basis. It is often the case
that a metric calculated from a basis via Eq. (\ref{eq:metric}) is
not automatically expressed in the simplest possible from and could
benefit from algebraic simplification.  Since later calculations
strongly depend on the form of the metric, it is usually worthwhile
to perform an explicit simplification of the components of
\texttt{g(dn,dn)} using the \grcmd{gralter} command. See Booklet
\grCalcRef, in particular the section on simplification strategies.}

If the basis calculation is requested, the algorithm used by
GRTensorIII to calculate the basis components of curvature tensors are
chosen so as to take maximal use of the basis formalism, ie. they
avoid the use of the metric.\footnote{In the particular case of the Ricci
tensor, the definition is in terms of the rotation coefficients:
\[
  R_{(a)(b)} := -\gamma_{(a)}{}^{(c)}{}_{(b),(c)}
    + \gamma_{(a)}{}^{(c)}{}_{(c),(b)}
    + \gamma^{(c)}{}_{(a)(d)}[ \gamma_{(b)}{}^{(d)}{}_{(c)}
      - \gamma_{(b)}{}^{(c)}{}_{(b)}]
    + \gamma_{(c)(a)(b)}\gamma^{(d)(c)}{}_{(d)}
    - \gamma_{(c)(a)(d)}\gamma^{(d)(c)}{}_{(b)}
\]
See Section \ref{sec:lib} for a definition of $\gamma_{(a)(b)(c)}$.}
%
However, if the tensor components have
already been calculated, the program will recognize this and simply
perform the conversion
\[
  R_{(a)(b)} = R_{cd} e_{(a)}{}^c e_{(b)}{}^d.
\]
(The opposite conversion is used when the basis components are known
and the tensor components are requested.)

The indices \texttt{pbdn} and \texttt{cbdn} specify the calculation of
the basis components of partial and covariant derivatives,
respectively. As described in the Booklet \grCalcRef, the command\\

\noindent\texttt{> grcalc ( R(bdn,bdn,pbdn) ):}\\

\noindent requests the calculation of
\[
  R_{(a)(b),(c)} := \frac{\partial R_{(a)(b)}}{\partial x^d}
    e_{(c)}{}^d,
\]
while the command\\

\noindent\texttt{> grcalc ( R(bup,bdn,cbdn) ):}\\

\noindent requests the calculation of the covariant derivative
\[
  R^{(a)}{}_{(b);(c)} := R_{(a)(b),(c)} + \gamma^{(a)}{}_{(b)(d)}
  R^{(d)}{}_{(c)} - \gamma^{(d)}{}_{(b)(c)} R^{(a)}{}_{(d)},
\]
defined in terms of the rotation coefficients,
\[
  \gamma_{(c)(a)(b)} := e_{(c)}{}^d e_{(a)d;e} e_{(b)}{}^e.
\]
(See Section \ref{sec:NP} for an alternate definition of this tensor.)
%
%------------------------------------------------------------------------------
\section{Bases in \grcmd{grdef}}\label{sec:def}
%------------------------------------------------------------------------------
%
Tensors and scalars which are not contained in the standard library
can be defined for a basis using the \grcmd{grdef} command, just as
they would in a metric space. The practical difference is that for
indexed objects the basis indices are enclosed in round braces,
\texttt{()}. Thus the basis components of the Ricci tensor,
$R_{(a)(b)}$, are referenced using the string
\begin{center}
  \texttt{R\{(a) (b)\}}\\
\end{center}
rather than
\begin{center}
  \texttt{R\{a b\}}\\
\end{center}
which refers to the tensor components, $R_{ab}$. Some other points are
also worth noting to make the use of \grcmd{grdef} more convenient.\\

In \textit{four dimensions} objects corresponding to each of the basis
vectors are defined by default when the spacetime is loaded or
created:
\begin{itemize}
  \item For contravariant null tetrads the objects \texttt{l(up)},
    \texttt{n(up)}, \texttt{m(up)}, and \texttt{mbar(up)} are assigned
    the values of $e_{(1)}{}^a \ldots e_{(4)}{}^a$ respectively. The
    vectors can be displayed as a group using the object name
    \texttt{nullt(up)}.  (For covariantly defined tetrads the \texttt{dn}
    versions of these vectors are assigned.)
  \item For general contravariant bases, the objects \texttt{e1(up)},
    \texttt{e2(up)}, \texttt{e3(up)}, and \texttt{e4(up)} are assigned
    the values of $e_{(1)}{}^a \ldots e_{(4)}{}^a$ respectively.  The
    vectors can be displayed as a group using the object name
    \texttt{basisv(up)}.
  \item For general covariant bases, the objects \texttt{w1(dn)},
    \texttt{w2(dn)}, \texttt{w3(dn)}, and \texttt{w4(dn)} are assigned
    the values of $e_{(1)a} \ldots e_{(4)a}$ respectively. The
    vectors can be displayed as a group using the object name
    \texttt{basisv(dn)}.
\end{itemize}

While the information contained in these objects is redundant (in the
sense that the components of the basis are already stored in the
object \texttt{e(bdn,up)}), they are included to facilitate new object
definitions in terms of these vectors. For instance, a
`purely-electric' electromagnetic tensor, $F_{ab}$, is defined by
\[
  F_{ab} := 4 \lambda l_{[a} n_{b]},
\]
which can be defined for GRTensorIII by using the command\\

\noindent\texttt{> grdef ( `F\{[a b]\} := 4*lambda*NPl\{[a\}*NPn\{b]\}` ):}\\

An important point is that the information contained in these
accessory objects (the vectors of \texttt{l(up)}, \texttt{e1(up)},
etc.) is copied from \texttt{e(bdn,up)} at the time the spacetime is
loaded or created, but not afterwards. Thus, if the components of
\texttt{e(bdn,up)} or \texttt{e(bdn,dn)} are modified (using
\grcmd{gralter}, for example) the modification will not be reflected
in the accessory objects. The opposite is also true: algebraic
simplification of the \texttt{l(up)}$\ldots$ \texttt{mbar(up)}, for
instance, will not be carried over to the corresponding components of
\texttt{e(bdn,up)}.

The GRTensorIII algorithms for the calculation of curvature tensors
rely on the versions of the basis components stored in the objects
\texttt{e(bdn,up)} and \texttt{e(bdn,dn)}. Thus, regardless of the
form of the accessory objects, for calculation purposes it should be
ensured that the components of \texttt{e(bdn,up)} and
\texttt{e(bdn,dn)} are in the simplest possible form.\\

The above discussion applies only to four dimensional spacetimes,
since the accessory basis vectors are not assigned automatically for any other
number of dimensions. In such cases, the Kronecker delta,
\texttt{kdelta}, can be used to isolate individual basis vectors.
For instance, the command\\

\noindent\texttt{> grdef (
  `mye3\{\^{}a\} := e\{(b) \^{}a\}*kdelta\{\^{}(b) \$3\}` ):}\\

\noindent assigns the third vector in \texttt{e(bdn,up)} to the object
\texttt{mye3(up)}:
\[
  e3^a := e_{(b)}{}^a \delta^{(b)}_3.
\]
The use of \texttt{kdelta} to isolate components is described more
fully in Booklet \grDefRef.

The basis components of derivatives are specified in object
definitions just as are the regular tensor derivatives. Thus, a comma,
`\texttt{,}', indicates a partial derivative, while a semi-colon,
`\texttt{;}', indicates a covariant derivative. Thus,
\[
  T_{(a)(b)(c)} := R_{(a)(b);(c)} + R_{(b)(c);(a)}
\]
could be defined using the command\\

\noindent\texttt{> grdef (
  `T\{(a)(b)(c)\} := R\{(a)(b);(c)\} + R\{(b)(c);(a)\}` ):}\\

\noindent The form of the following index will indicate which 
components (either basis or tensor) of the derivative are to be
calculated. The operations specified by each type of derivative are
listed in Booklet \grDefRef.\\

Finally, note that for null tetrads, a set of derivatives operators
are defined corresponding to partial derivatives along the basis
vectors. The operators
\[
  D := l^a\partial_a, \qquad \Delta := n^a\partial_a, \qquad
  \delta := m^a\partial_a, \qquad \delta^* := \bar{m}^a\partial_a,
\]
are defined by the GRTensorIII operators
\begin{center}
  \texttt{Dl[]}, \qquad \texttt{Dn[]}, \qquad \texttt{Dm[]}, \qquad
  \texttt{Dmbar[]},
\end{center}
respectively. The operators act on any GRTensorIII object placed in
the square braces.\footnote{GRTensorIII operators are described in
Booklet \grCalcRef.} For example, the right-hand side of the
NP equation
\[
  \delta\tau - \Delta\sigma - (\mu\sigma + \lambda^*\rho)
    - \tau (\tau+\beta-\alpha^*) + \sigma (3\gamma-\gamma^*)
    + \kappa\nu^* - \Phi_{02} = 0,
\]
could be defined using the command\\

\noindent\texttt{> grdef ( `NPeq1 := Dm[NPtau] - Dn[NPsigma]
  - (NPmu*NPsigma + NPlambdabar*NPrho) \\
  \indent - NPtau*(NPtau + NPbeta - NPalphabar)
  + NPsigma*(3*NPgamma - NPgammabar) + \\
  \indent NPkappa*NPnubar - Phi02` ):}\\

\noindent (See Section \ref{sec:NP} for definitions of the objects
\texttt{NPtau}, \texttt{NPsigma}, etc.)
%
%------------------------------------------------------------------------------
\section{The standard library} \label{sec:lib}
%------------------------------------------------------------------------------
%
The following sections list the objects contained within the standard
GRTensorIII library which are defined specifically for use with bases.
They can be calculated using \grcmd{grcalc} once a basis has been
created or loaded into the current session (see Booklet \grMakegRef).

Objects not present in the following lists can be defined using the
\grcmd{grdef} command, as described in the previous section. A further
list of tensors calculable from a metric is given in Booklet
\grCalcRef.
%
%------------------------------------------------------------------------------
\subsection*{General bases}
%------------------------------------------------------------------------------
%
Definitions listed in the following table are for objects which have
not already been defined in the standard library list in Booklet
\grCalcRef. Basis definitions of the objects given in that table
can be accessed by choice of indices. For instance, the basis components
of the Riemann tensor, $R^{(a)}{}_{(b)(c)(d)}$ are calculated using
the command\\

\noindent\texttt{> grcalc ( R(bup,bdn,bdn,bdn) ):}\\

\renewcommand{\baselinestretch}{1.5}\normalsize
\begin{longtable}[c]{lp{.6\textwidth}}\hline\hline
GRTensorIII name & Definition\\ \hline
\texttt{e(bdn,up)}	& basis vectors, $e_{(a)}{}^b$ \\
\texttt{eta(bdn,bdn)}	& basis inner product, $\eta_{(a)(b)}$ \\
\texttt{basisv(up)}	& collection of basis vectors,
			  $e_{(1)}{}^a \ldots e_{(4)}{}^a$ \\
\texttt{lambda(bdn,bdn,bdn)} & $\lambda_{(a)(b)(c)} := 
			  e_{(b)[i,j]} e_{(a)}{}^d e_{(c)}{}^e$ \\
\texttt{rot(bdn,bdn,bdn)} & rotation coefficients, \\
			& $\gamma_{(a)(b)(c)} := \frac{1}{2} \left(
			  \lambda_{(a)(b)(c)}+\lambda_{(c)(a)(b)} - 
			  \lambda_{(b)(c)(a)} \right)$ \\
\texttt{str(bdn,bdn,bdn)} & structure constants, \\
			& $C_{(a)(b)(c)} := \gamma_{(a)(c)(b)}
			  - \gamma_{(a)(b)(c)}$ \\
\hline
\end{longtable}
\renewcommand{\baselinestretch}{1}\normalsize

Four dimensional spacetimes which are specified in the form of a set
of \textit{contravariant} basis vectors additionally have the objects
\texttt{e1(up)}, \texttt{e2(up)}, \texttt{e3(up)}, and \texttt{e4(up)}
assigned, corresponding to the four basis vectors, $e_{(a)}{}^b$. If
the spacetime was specified by a set of \textit{covariant} basis
vectors, then the objects \texttt{w1(dn)}, \texttt{w2(dn)},
\texttt{w3(dn)}, and \texttt{w4(dn)} are automatically assigned.
For display purposes, the vectors can be accessed as a group using
the object \texttt{basisv(up)}.
See Section \ref{sec:def} for a description of these objects.
%
\pagebreak
%------------------------------------------------------------------------------
\subsection*{The Newman-Penrose formalism}\label{sec:NP}
%------------------------------------------------------------------------------
%
In addition to the objects specified above, a number of special
quantities can be calculated for geometries specified by a null
tetrad. The following objects are defined in
\cite{newman/penrose:1962}.\\

\begin{longtable}[c]{p{.4\textwidth}p{.4\textwidth}}\hline\hline
GRTensorIII name & Definition\\ \hline
& \\
\texttt{l(up)}, \texttt{n(up)}, \texttt{m(up)}, \texttt{mbar(up)}
				& basis vectors,\footnote{As explained
				in Section \ref{sec:def}, the components
				of these basis vectors are copied from
				\texttt{e(bdn,up)} when the spacetime is
				loaded.}\\
				& $l^a, n^a, m^a, \bar{m}^a$. \\
\texttt{nullt(up)}			& collection of basis vectors, \\
				& $\{ l^a, n^a, m^a, \bar{m}^a \}$. \\
\texttt{testNP(bdn,bdn)}	& confirms NP inner product, \\
				& $testNP_{(a)(b)} := e_{(a)c}e_{(b)}{}^c$.\\
& \\
\texttt{NPkappa, NPsigma, NPlambda, NPnu} & spin coefficients, \\
\texttt{NPrho, NPmu, NPtau, NPpi,} & $\kappa, \sigma, \ldots, \beta$\\
\texttt{NPepsilon, NPgamma, NPalpha, NPbeta} & \\
& \\
\texttt{NPkappabar, NPsigmabar,} & spin coefficients\\
\texttt{NPlambdabar, NPnubar, NPrhobar,} & (complex conjugates) \\
\texttt{NPmubar, NPtaubar, NPpibar,} & $\bar{\kappa}, \bar{\sigma},
				\ldots \bar{\beta}$ \\
\texttt{NPepsilonbar, NPgammabar, NPalphabar, NPbetabar} & \\
& \\
\texttt{Phi00, Phi01, Phi02, Phi10,} & Ricci scalars, \\
\texttt{Phi11, Phi12, Phi20, Phi21}  & $\Phi_{00}, \Phi_{01}, \ldots,
				  \Phi_{22}, \Lambda$ \\
\texttt{Phi22, Lambda} & \\
& \\
\texttt{Psi0, Psi1, Psi2, Psi3, Psi4} & Weyl scalars,
				$\Psi_0, \ldots, \Psi_4$\\
& \\
\texttt{Petrov} 		& Petrov type \\
& \\
\texttt{NPspin}			& collection of spin coefficients,\\
				& $\{\kappa,\ldots,\beta\}$ \\
\texttt{NPspinbar}		& collection of spin coefficients\\
				& $\{\bar{\kappa},\ldots,\bar{\beta}\}$ \\
\texttt{RicciSc}		& collection of Ricci scalars, \\
				& $\{\Phi_{00},\ldots,\Phi_{22},\Lambda\}$ \\
\texttt{WeylSc}			& collection of Weyl scalars, \\
				& $\{\Psi_0,\ldots,\Psi_4\}$\\
\hline
\end{longtable}

A set of four derivative operators, corresponding to partial derivatives
along the basis vectors, are also defined for the Newman-Penrose formalism.
In the following example the argument `\grarg{object}' represents an
arbitrary GRTensorIII object on which the operator is to act.\\

\renewcommand{\baselinestretch}{1.5}\normalsize
\begin{longtable}[c]{ll}\hline\hline
GRTensorIII name & Definition\\ \hline
\texttt{Dl[\grarg{object}]}	& 
		$D  := l^c\partial_c$ \\
\texttt{Dn[\grarg{object}]}	&
		$\Delta := n^c\partial_c$\\
\texttt{Dm[\grarg{object}]}	&
		$\delta := m^c\partial_c$\\
\texttt{Dmbar[\grarg{object}]}	&
		$\delta^* := \bar{m}^c\partial_c$\\
\hline
\end{longtable}
\renewcommand{\baselinestretch}{1}\normalsize
%
%------------------------------------------------------------------------------
\subsection*{Alternate NP definitions}\label{sec:altNP}
%------------------------------------------------------------------------------
%
An alternate set of definitions for the Newman-Penrose coefficients is
provided by Allen, et al. \cite{allen/etal:1994}. The definitions
aim to gain a computational advantage by avoiding the inversion of
the basis vectors (that is, only the covariant components are needed).
These algorithms have been incorporated into GRTensorIII and tested
with results presented in \cite{pmsl96}. It was found that no consistent
computational advantage was gained through use of the alternative
formulas, though in certain cases the covariant
definitions do outperform the standard NP formulas. The alternate
NP algorithms can be accessed by requesting the calculation of the
objects listed in the table below. These names of these objects correspond
to the NP names listed in the previous section prefixed by a `\texttt{C}'.
Thus the command\\

\noindent\texttt{> grcalc ( WeylSc ):}\\

\noindent requests the calculation of the Weyl curvature coefficients using
the formulas defined by Newman-Penrose, while the command\\

\noindent\texttt{> grcalc ( CWeylSc ):}\\

\noindent calculates the same objects using the alternate `covariant'
formulas.\footnote{Note that if one of the above commands is executed
in a given session, then GRTensorIII will recognize the Weyl
coefficients as having been calculated. Thus the use of the second
command will return the previously calculated results. In order to
compare the results of each command, they must be issued individually
in separate GRTensorIII sessions.}

\begin{longtable}[c]{p{.4\textwidth}p{.4\textwidth}}\hline\hline
GRTensorIII name & Definition\\ \hline
& \\
\texttt{Ckappa, Csigma, Clambda, Cnu} & spin coefficients, \\
\texttt{Crho, Cmu, Ctau, Cpi,} & $\kappa, \sigma, \ldots, \beta$\\
\texttt{Cepsilon, Cgamma, Calpha, Cbeta} & \\
& \\
\texttt{Ckappabar, Csigmabar,} & spin coefficients\\
\texttt{Clambdabar, Cnubar, Crhobar,} & (complex conjugates) \\
\texttt{Cmubar, Ctaubar, Cpibar,} & $\bar{\kappa}, \bar{\sigma},
				\ldots \bar{\beta}$ \\
\texttt{Cepsilonbar, Cgammabar, Calphabar, Cbetabar} & \\
& \\
\texttt{CPhi00, CPhi01, CPhi02, CPhi10,} & Ricci scalars, \\
\texttt{CPhi11, CPhi12, CPhi20, CPhi21}  & $\Phi_{00}, \Phi_{01}, \ldots,
				  \Phi_{22}, \Lambda$ \\
\texttt{CPhi22, CLambda} & \\
& \\
\texttt{CPsi0, CPsi1, CPsi2, CPsi3, CPsi4} & Weyl scalars,
				$\Psi_0, \ldots, \Psi_4$\\
& \\
\texttt{Cspin}			& collection of spin coefficients,\\
				& $\{\kappa,\ldots,\beta\}$ \\
\texttt{Cspinbar}		& collection of spin coefficients\\
				& $\{\bar{\kappa},\ldots,\bar{\beta}\}$ \\
\texttt{CRicciSc}		& collection of Ricci scalars, \\
				& $\{\Phi_{00},\ldots,\Phi_{22},\Lambda\}$ \\
\texttt{CWeylSc}		& collection of Weyl scalars, \\
				& $\{\Psi_0,\ldots,\Psi_4\}$\\
\hline
\end{longtable}
%
%------------------------------------------------------------------------------
\bibliographystyle{unsrt}
\bibliography{grtensor}
%------------------------------------------------------------------------------
\vspace*{\fill}
\large
\noindent The information contained in this booklet is also available from the
following online help pages:\\

\noindent\texttt{?grt\_basis}, \texttt{?grt\_objects},
\texttt{?grt\_operators}, \texttt{?grcalc}, \texttt{?grdef}, \texttt{?petrov}.
%
%------------------------------------------------------------------------------
\end{document}
%==============================================================================
