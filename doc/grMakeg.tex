%=================================================================-*-LaTeX-*-==
% GRTensorIII 1.50 Manual
% Booklet B: Specifying spacetimes
%
% Denis Pollney
% July 1996
%==============================================================================
\documentclass{article}
\usepackage{amsmath}
%\usepackage{maple2e}
\usepackage{grtensor}
\usepackage{grbooks}
%------------------------------------------------------------------------------
\begin{document}
\grlabel{\grMakegLabel}
\grtitle{\grMakegTitle}
\grdate{Nov 2016}
\grtitlepage
%==============================================================================
% Body.
%==============================================================================
\copyrightpage
\noindent The simplest way to specify a spacetime geometry in GRTensorIII
is to use the \grcmd{makeg} facility. This function aids input by
prompting the user for the information required to specify a
coordinate metric (a $n\times n$ dimensional 2-tensor) or basis (a set
of $n$ linearly independent vectors related by a user-defined inner
product).\footnote{Although GRTensorIII allows specification of
spacetimes as both metrics and bases, in these booklets we will often
refer to both formats as simply \textit{metrics} with the
understanding that we are referring also to bases created via
\grcmd{makeg}.}

In addition to using \grcmd{makeg}, new spacetimes can be constructed
from previously defined spacetimes through the use of the commands:
\begin{center}
  \begin{tabular}{rl}
    \grcmd{grtransform} & -- performs a coordinate transformation of a 
      metric,\\
    \grcmd{nptetrad} & -- constructs a null tetrad corresponding to a metric,\\
    \grcmd{nprotate} & -- performs rotations of a null tetrad, and\\
    \grcmd{grnewmetric} & -- creates a metric from a $2\times 2$ MapleV array.
  \end{tabular}
\end{center}
These commands are described in Sections 
\ref{sec:grtransform}--\ref{sec:grnewmetric} of this booklet.

The metrics created for use in GRTensorIII can be saved to an ASCII
file directly using \grcmd{makeg} or the \grcmd{grsaveg}
command. These files can be loaded into GRTensorIII using either the
\grcmd{qload} or \grcmd{grload} commands. A directory of commonly used
metric/basis files is available from the GRTensor world-wide-web pages
\cite{www}.

%-----------------------------------------------------------------------------
\section{Metrics and bases}
%-----------------------------------------------------------------------------
The \grcmd{makeg} can be used to enter all of the information needed
to specify a spacetime, either as a metric or set of basis
vectors. This section establishes some notation that will be used
throughout these booklets.

Metrics are $n$ dimensional 2-tensors which are assumed to
be symmetric:
\[
  g_{ab} :=
  \left[
    \begin{array}{cccc}
      g_{11} & g_{12} & \cdots & g_{1n} \\
      g_{12} & g_{22} &        & \vdots \\
      \vdots &        & \ddots &        \\
      g_{1n} & \cdots &        & g_{nn}
    \end{array}
  \right]
\]
The components $g_{ij}$ are functions of the $n$ coordinates
$\{ x_1, \ldots, x_n\}$.

Bases are sets of $n$ independent vectors,
\[
  \left\{
    e_{(1)}{}^a = [ e_{11}, \ldots, e_{1n} ], \quad \ldots \quad
    e_{(n)}{}^a = [ e_{n1}, \ldots, e_{nn} ],
  \right\}
\]
whose inner product is defined by the symmetric matrix
\[
  \eta^{(a)(b)} =
    \left[
      \begin{array}{cccc}
      \eta_{11} & \eta_{12} & \cdots & \eta_{1n} \\
      \eta_{12} & \eta_{22} &        & \vdots \\
      \vdots    &           & \ddots &        \\
      \eta_{1n} & \cdots    &        & \eta_{nn}
    \end{array}
  \right],
\]
via
\[
  \langle e_{(i)}{}^a, e_{(j)}{}^b\rangle 
    = \eta^{(i)(j)} e_{(i)}{}^a e_{(j)}{}^b
    = g^{ab}.  
\] 
(Note that here and throughout GRTensorIII we use the convention that
basis vectors are labeled with bracketed indices $\{ (a), (b), \ldots \}$.)
%
%-----------------------------------------------------------------------------
\section{Using \grcmd{makeg}}\label{makeg}
%-----------------------------------------------------------------------------
%
The syntax for the \grcmd{makeg} command is as follows:\\
\begin{cmdspec}
  \label{spec:makeg}
  \grcmdline{makeg ( \grarg{metricName}, [\grarg{metricPath}] )}

  \begin{description}

  \item[\grarg{metricName}] -- the name to be given to the newly
    created metric or basis. This name will also be used to create the
    filename (with a \texttt{.mpl} extension) in which the metric
    information will be saved.  

\item[\grarg{metricPath}] -- \textit{(optional)} this argument can 
    be used to specify a directory name in which the metric file is to
    be placed. The argument is a string with directory levels
    separated by forward slashes, `\texttt{/}'. If this argument is
    not specified the directory stored in the global option variable
    \texttt{grOptionMetricPath} is used instead.
  \end{description}

  \grexample{makeg ( bondi )}
\end{cmdspec}

Upon issuing the \grcmd{makeg} command, the user is prompted to
enter all of the information necessary to specify the spacetime.
The first of these prompts asks for the format in
which the data is to be entered:\footnote{Note 
that the first option refers to `\texttt{g(dn,dn)}'. This is the standard
GRTensorIII representation of the covariant metric. See Booklet \grCalcRef~
for more details regarding this notation.}
\begin{ttfamily}
  \begin{tabbing}
    Do you wish to enter a \= 1) metric [g(dn,dn)], \\
                           \> 2) line element [ds], \\
                           \> 3) non-holonomic basis [e1...e4], or \\
                           \> 4) NP tetrad [l,n,m,mbar]
  \end{tabbing}
\end{ttfamily}
The reply should be an integer from 1 to 4. Each option is described in turn
below.
%
%------------------------------------------------------------------------------
\subsection{The metric as a covariant 2-tensor} \label{sec:metricentry}
%------------------------------------------------------------------------------
%
The user is prompted to enter the following information:
\begin{description}
  \item[Coordinates] These are entered as a MapleV list,
    eg. \texttt{[r,theta,phi,t]}. The names which are used as coordinates
    must be previously unassigned. Any number of coordinates may be entered,
    and their number will determine the dimension of the spacetime.
  \item[Signature] This is an integer, $s = n_+ - n_-$, where $n_+$ is
    the number of positive components on the diagonal of the metric, and
    $n_-$ is the number of negative components on the diagonal of the
    metric in a locally orthonormal basis. If the global variable
    \texttt{grOptionLLSC} is set to \texttt{true}, then this prompt does
    not appear.\footnote{See the discription of the \texttt{grOptionLLSC}
    in Booklet \grSetupRef~and the \texttt{?groptions} online help page.}
  \item[Form of the metric] (diagonal or symmetric) This will reduce the
    number of coordinates which the user is required to input. If the
    `diagonal' option is chosen then off-diagonal components are automatically
    set to zero. Components on the lower diagonal are automatically
    set equivalent to those on the upper diagonal.  GRTensorII
    does not currently handle non-symmetric metrics.
  \item[Metric components] A prompt appears for each unknown component of
    the covariant metric. Keep in mind that if a component involves functions
    of the coordinates then the coordinate dependence must be stated
    explicitly, as in \texttt{M(r,t)} for example.
\end{description}
%
%------------------------------------------------------------------------------
\subsection{The metric as a line-element}
%------------------------------------------------------------------------------
%
It is often most convenient to enter a metric in the form that it is
most commonly presented in journals and texts, ie. as a line element.
This helps to minimize the risk of transcription errors between paper and
the computer.

Choosing the second option from the \grcmd{makeg} menu, the user is
first prompted to enter the coordinate names. As above, this should
take the form of a MapleV list of unassigned names,
eg. \texttt{[r,theta,phi,t]}.

The line element is then entered using the notation \texttt{d[x]} to 
represent the coordinate differential $dx$. For example, the line element
\[
  dx^2 + (dy + dz)^2
\]
would be entered as
\begin{center}
  \begin{texttt}
    d[x]\^{}2 + (d[y] + d[z])\^{}2:
  \end{texttt}
\end{center}
Naturally, the line element must be a quadratic form in the coordinate
differentials.

Once the line element is entered, it is converted to a $n\times n$
covariant 2-tensor (\texttt{g(dn,dn)} in GRTensorII's notation) which
is displayed so that they can be checked for errors in the input.
%
%------------------------------------------------------------------------------
\subsection{Non-holonomic bases}
%------------------------------------------------------------------------------
%
Choosing Options 3 or 4 from the \grcmd{makeg} menu allows one to
enter the components of a non-holonomic basis. The first of these
options (option \texttt{3: non-holonomic basis [e1...e4]}) allows the
user to also specify the inner product, $\eta^{(a)(b)}$, between
individual basis vectors. The second (option \texttt{4: NP tetrad
[l,n,m,mbar]}) assumes the inner product of a Newman-Penrose
null tetrad,
\begin{equation}
  \eta^{(a)(b)} :=
    \left[
      \begin{array}{cccc}
        0 & 1 & 0 & 0 \\
        1 & 0 & 0 & 0 \\
        0 & 0 & 0 &-1 \\
        0 & 0 &-1 & 0
      \end{array}
    \right].
  \label{eq:nullip}
\end{equation}

In both cases, the user is first prompted to input the coordinate names
as a MapleV list, eg. \texttt{[r,theta,phi,t]}.\\

The next prompt asks if covariant or contravariant components of the
tetrad are to be entered.\footnote{The option to enter both forms of
tetrad also exists, since in certain cases the inversion of the tetrad
introduces complicated terms (especially radicals) which MapleV has
difficulty simplifying. In such cases it is sometimes preferable to
enter both forms of the vector if they are known in a simple form. The
user must be careful that the forms are mutually consistent. (For
NP-tetrads, the object \texttt{testNP(bdn,bdn)} can be calculated to ensure
that at least the basis vectors satisfy an NP inner product.} The user
is then asked to enter the basis vectors.  These should be specified
in the form of a MapleV list, eg. \texttt{[1,0,0,0]}.  The number of
vectors that are entered is determined by the number of coordinates
making up the spacetime. For general bases, the vectors are labeled
by the numbers $1, \ldots, n$. For null tetrads, the basis vectors are
labeled \texttt{l}, \texttt{n}, \texttt{m}, \texttt{mbar}
respectively.\\

If the user has chosen to enter a general basis (Option 3 from the
\grcmd{makeg} starting menu), the next prompts ask for the components
of the inner product between the basis vectors. This is the
contravariant two-tensor $\eta^{(a)(b)}$ whose $(a)(b)$ component is
the value of the inner product of basis vectors $a$ and $b$.  In
common applications, this tensor will have constant coefficients. The
curvature tensors defined in the GRTensorIII standard object libraries
permit the use of inner products with non-constant coefficients,
however it is required that the inner product be symmetric across the
diagonal.

If a null tetrad has been selected (Option 4 from the starting menu),
then the inner product between the basis vectors is assumed to be of the
form specified in Eq. (\ref{eq:nullip}).\\

Once the inner product is selected (or chosen by default in the case
of null tetrads) the basis vectors and inner product are both
displayed so that they can be checked for errors in the input.
%
%------------------------------------------------------------------------------
\subsection{Other options}
%------------------------------------------------------------------------------
%
Once the data needed to specify the spacetime has been entered, a menu presents
a number of additional options for correcting, saving, and supplementing the
data. Typically the menu for metric entry looks like:
\begin{ttfamily}
  \begin{tabbing}
    You may choose to \= 0) Use the metric without saving it, \\
                      \> 1) Save the metric as it is, \\
                      \> 2) Correct an element of the metric, \\ 
                      \> 3) Re-enter the metric, \\
                      \> 4) Add/change constraint equations, \\ 
                      \> 5) Add a text description, or \\
                      \> 6) Abandon this metric and return to Maple.
  \end{tabbing}
\end{ttfamily}
The options in this menu have obvious analogues for the cases of basis or
null-tetrad input. Each option is described briefly below:

\begin{description}
  \item[Use the metric without saving it:]
    The metric is initialized in the current session as \texttt{g(dn,dn)},
    and can thus be displayed via the command \texttt{grdisplay(g(dn,dn))}.
    The components are not saved to a file, and so will be lost when the
    MapleV session is ended. The components of the metric can be saved at
    a later time using the \grcmd{grsaveg} command.
%
  \item[Save the metric as it is:]
    Any information that has been entered in this invocation of \grcmd{makeg}
    is saved to the file `\grarg{metricName}\texttt{.mpl}' 
    (where \grarg{metricName} is the name specified
    in the argument to \grcmd{makeg}) and in the directory given by the global
    variable \texttt{grOptionMetricPath} or by the optional 
    \grarg{metricPath} argument to \grcmd{makeg}.\footnote{For a description
    of the \texttt{grOptionMetricPath} variable, see Booklet \grSetupRef~or
    the online help page \texttt{?groptions}.}
    The metric is initialized for the current session as in the previous
    option.
%
  \item[Correct an element of the metric:]
    The user is prompted to enter the index values of the metric component 
    to be corrected. This should take the form of a two-component MapleV list
    which contains coordinate names, eg. \texttt{[r,theta]}.
%
  \item[Re-enter the metric:]
    The user is prompted to re-enter each component of the metric in
    succession as described in Section \ref{sec:metricentry} above.
%
  \item[Add/change constraint equations:]
    The user has the opportunity to add information taking the form of
    constraint equations to the metric. For example, consider a
    metric containing the function $m(r,t)$ which is required to satisfy the 
    partial differential equations
    \[
      \frac{\partial}{\partial r}m(r,t) = r^2 m(r,t), \quad \text{and} \quad
      \frac{\partial}{\partial t}m(r,t) = t^2 m(r,t)^2.
    \]
    Choosing this option, the user is prompted to enter these constraint
    equations as a list:
    \vspace{\baselineskip}\\
      \texttt{
        [ diff ( m(r,t), r ) = r\^{}2*m(r,t), 
        diff ( m(r,t), t ) = t\^{}2*m(r,t)\^{}2 ]
      }
    \vspace{\baselineskip}\\
    These constraints could then be applied to objects calculated from this
    metric via \grcmd{grcalc} by using the appropriate options from \
    \grcmd{gralter}.
   (See Section \ref{sec:constraint} and Booklet \grCalcRef.)
%
  \item[Add a text description:]
    A line of text describing the metric can be saved along with its components
    so that it may be more easily identified later. Such a note might include
    its full name, a journal reference, or some descriptive
    adjectives.\footnote{
    Inclusion of such descriptions is strongly recommended as it can make
    large directories of metrics much more manageable. For instance, in Unix
    systems, some idea of the contents of each file in a metrics directory
    can be obtained using the command `\texttt{grep Info *.mpl}'.
    This command will list the text descriptions of each metric file (see
    Section \ref{sec:metricfiles}).}
\end{description}

The new metric created by \grcmd{makeg} automatically becomes the default
metric to be used for subsequent calculations (see Section \ref{sec:default}).
%
%------------------------------------------------------------------------------
\section{Loading spacetimes from a file} \label{qload}
%------------------------------------------------------------------------------
%
Metrics and bases created by \grcmd{makeg} are saved automatically to
metric files in the default directory specified by the
\texttt{grOptionMetricPath} variable. These metric files can be loaded
in future sessions using \grcmd{qload} or \grcmd{grload}. Each of
these is described below.\\

A directory of commonly used metric files is distributed with
GRTensorII.  A collection is also kept online \cite{www}. The format
of metric files is outlined in Section \ref{sec:metricfiles}.\\
%
\begin{cmdspec}
  \label{spec:qload}
  \grcmdline{qload ( \grarg{metricName} )}

  \begin{description}
    \item[\grarg{metricName}] -- the name of the metric to be loaded.
  \end{description}

  \grexample{qload ( schw )}
\end{cmdspec}

The \grcmd{qload} command searches the directories specified by the
\texttt{grOptionqloadPath} global variable for the file
\grarg{metricName}\texttt{.mpl}. If \texttt{grOptionqloadPath} is not
assigned, or if the specified file is not found by one of the directories
specified by \texttt{grOptionqloadPath}, then the directory specified
by the global variable \texttt{grOptionMetricPath} is searched.\\
%
\begin{cmdspec}
  \label{spec:grload}
  \grcmdline{grload ( \grarg{metricName}, \grarg{metricFile} )}

  \begin{description}
    \item[\grarg{metricName}] -- the name by which the new metric is to be
      referenced in the current session.
    \item[\grarg{metricFile}] -- the complete file name (including directory
      path and `\texttt{.mpl}' extension) of the metric file to be loaded.
  \end{description}

  \grexample{grload ( schwarzschild, `c:/mydir/metrics/schw.mpl` )}
\end{cmdspec}

The \grcmd{grload} command loads the file specified by the \grarg{metricFile}
command. For the remainder of the current session, the metric is referenced
by the \grarg{metricName} parameter, which (unlike \grcmd{qload}'s) need not
be related to the name of the metric file.\\

A spacetime loaded by \grcmd{qload} or \grcmd{grload} becomes the default
metric to be used for subsequent calculations
(see Section \ref{sec:default}).\\

The following objects are initialized by the loading commands (as well
as by \grcmd{makeg}), depending on the form of spacetime specified by
the input file (ie. metric or basis):\footnote{In addition to the
initialization of these objects, the following assumptions regarding
the signature are made when loading a four-dimensional spacetime when
the global \texttt{grOptionLLSC} variable is set \texttt{true}. If the
spacetime is specified in the form of a metric, $g_{ab}$, or as a
basis with a general inner product, then the signature is by default
set to +2.  If the spacetime is specified as a null tetrad satisfying
an NP inner product, Eq. (\ref{eq:nullip}), then the signature is set
to -2. The assumed signature of the spacetime is stored in the
\texttt{sig} object and can be viewed by using the command
\texttt{grdisplay(sig)}.
 In this version of GRTensorIII, the value of the signature
is only used in generating an NP tetrad via the \grcmd{nptetrad}
command (see Section \ref{sec:nptetrad}, below). If the \texttt{grOptionLLSC}
variable is set to \texttt{false}, then \texttt{sig} is not initialized
unless the signature is explicitly given in the metric file.}
\begin{center}
  \begin{tabular}{rl}
    metric: & \texttt{ds}, \texttt{g(dn,dn)} \\
    general covariant basis: & \texttt{eta(up,up)}, \texttt{e(bdn,dn)},
      \texttt{basis(dn)} \\
    general contravariant basis: & \texttt{eta(up,up)}, \texttt{e(bdn,up)},
      \texttt{basis(up)} \\
    covariant null tetrad: & \texttt{eta(up,up)}, \texttt{e(bdn,dn)},
      \texttt{nullt(dn)} \\
    contravariant null tetrad: & \texttt{eta(up,up)}, \texttt{e(bdn,up)},
      \texttt{nullt(up)} \\
  \end{tabular}
\end{center}
Additionally, for all of the above types of spacetime the objects
\begin{center}
  \texttt{x(up)}, \texttt{dimension},
\end{center}
are also initialized, as well as
\begin{center}
  \texttt{Info}, \texttt{constraint}, \texttt{sig},
\end{center}
if applicable.\\

See Booklet \grCalcRef~or the online help pages \texttt{?grt\_objects}
and \texttt{?grt\_basis} for a description of these objects. The
global variables \texttt{grOptionqloadPath} and
\texttt{grOptionMetricPath} are described in Booklet \grSetupRef~and
the \texttt{?groptions} online help page.

Once a metric name has been used in a session, it can not be
re-used. For instance, if the \texttt{schw} spacetime has already been
loaded, attempts to use the commands \texttt{makeg(schw)} or
\texttt{qload(schw)} will fail. Metrics can be cleared from a session
using the \grcmd{grclear} command, described in Booklet \grCalcRef.
%
%------------------------------------------------------------------------------
\subsection{The default spacetime} \label{sec:default}
%------------------------------------------------------------------------------
%
A number of commands described in this booklet (\grcmd{makeg},
\grcmd{qload}, \grcmd{grtransform}, etc.) create new metrics or bases
which describe a background geometry for which tensors can be
calculated. Generally, once these commands are issued, the spacetime
which they specify becomes the \textit{default} spacetime (or
background geometry) for future calculations. For instance, once the
\mbox{\texttt{qload(schw)}} command is issued (the command to load
the Schwarzschild metric), all subsequent calculations will assume
this background metric until the default is changed.\\

If a number of different metrics have been created or loaded in single
GRTensorIII section, one can switch between them by using the \grcmd{grmetric}
command, which has the form:\\
%
\begin{cmdspec}
  \label{spec:grmetric}
  \grcmdline{grmetric ( \grarg{metricName} )}

  \begin{description}
    \item[\grarg{metricName}] -- the name of a metric (or basis) which has been
      created or previously loaded in the current session.
  \end{description}

  \grexample{grmetric ( schw )}
\end{cmdspec}

This command makes the metric named in its argument the default metric
for subsequent calculations.\\

Alternatively, one can perform calculations using metrics other than
the default metric by specifying the metric name as a parameter to the
objects being calculated. The metric name is placed in square brackets
following the name of the object. For instance,\\

\noindent\texttt{> grcalc ( R[schw](dn,dn) ):}\\

\noindent calculates the covariant Ricci tensor for the \texttt{schw}
spacetime, even if that spacetime is not the current default. See the
Booklet \grCalcRef~for details on the use of the \grcmd{grcalc}
command.
%
%------------------------------------------------------------------------------
\section{Coordinate transformations} \label{sec:grtransform}
%------------------------------------------------------------------------------
%
The command \grcmd{grtransform} can be used to perform coordinate
transformations of the metric tensor. It takes the following form:\\
%
\begin{cmdspec}
  \label{spec:grtransform}
  \grcmdline{grtransform ( \grarg{oldmetric}, \grarg{newmetric},
    \grarg{xform}, [\grarg{newsig}] )}

  \begin{description}
    \item[\grarg{oldmetric}] -- the name of the metric to be transformed.  
    \item[\grarg{newmetric}] -- the name of by which the transformed metric
      is to be referenced in the future.
    \item[\grarg{xform}] -- a list specifying the transformation.
    \item[\grarg{newsig}] -- (\textit{optional}) an integer indicating
      the signature of the transformed metric, if different from the
      original signature.
  \end{description}

  \grexample{grtransform ( schw, kruskal, xform := [\\
    u(r,t) = sqrt(r/(2*m)-1)*exp(r/(4*m))*cosh(t/(4*m)),\\ 
    v(r,t) = sqrt(r/(2*m)-1)*exp(r/(4*m))*sinh(t/(4*m)),\\
    Theta(theta) = theta, Phi(phi) = phi ] )}
\end{cmdspec}

The third argument, \grarg{xform}, is specified by a list of functions giving
the old coordinates in terms of the new coordinates or vice versa. For
instance if we are performing the transformation from coordinates
$(t,x,y,z)$ to new coordinates $(\tau,u,v,w)$, 
then we could specify \grarg{xform} as a set of functions of old in terms
of new,\\

\noindent\texttt{xform := [ t(tau,u,v,w)= $\ldots$, x(tau,u,v,w)= $\ldots$,
  y(tau,u,v,w)= $\ldots$, z(tau,u,v,w) = $\ldots$ ]:}\\

\noindent or new in terms of old,\\

\noindent\texttt{xform := [ tau(t,x,y,z)= $\ldots$, u(t,x,y,z)= $\ldots$,
  v(t,x,y,z)= $\ldots$, w(t,x,y,z)= $\ldots$ ]:}\\

\noindent Note that for an $n$-dimensional spacetime, $n$ functions must be 
specified.  Also, none of the new set of coordinates can be the same
as any of the old coordinates. For instance, if both spacetimes are
spherically symmetric, new coordinate names for the angles
$( \theta, \phi )$ must be given for the new spacetime, such as
$( \Theta, \Phi )$.

The following set of commands performs the transformation of the Schwarzschild
spacetime from spherical to Kruskal coordinates:\footnote{For a more complete
presentation of this example see the demonstration file \texttt{kruskalo.ms},
which can be downloaded from the GRTensorIII world-wide-web pages \cite{www}.}

\begin{ttfamily}
  \begin{tabbing}
    >  qload ( schw ):\vspace{\baselineskip}\\

    >  xform := [ \= u(r,t) = sqrt(r/(2*m)-1)*exp(r/(4*m))*cosh(t/(4*m)),\\
              \>  v(r,t) = sqrt(r/(2*m)-1)*exp(r/(4*m))*sinh(t/(4*m)),\\
              \>  Theta(theta) = theta,\\
              \>  Phi(phi) = phi ]:\\

    >  grtransform ( schw, kruskal, xform ):
  \end{tabbing}
\end{ttfamily}

It is not necessary that each of the transformation functions be specified 
as a function of all $n$ variables, but each of the variables must be
represented in at least one of the functions of the set. For instance,
in the example above, at least one of the components of \texttt{xform}
must be a function of $t$, at least one must be a function of $r$, etc.

The transformed coordinates can be saved to a file using the
\grcmd{grsaveg} command (see Section~\ref{sec:grsaveg}). The new metric
created by the \grcmd{grtransform} command becomes the new default
metric for subsequent calculations (see Section \ref{sec:default}).
%
%------------------------------------------------------------------------------
\section{Constraint equations} \label{sec:constraint}
%------------------------------------------------------------------------------
%
The \grcmd{makeg} command provides the ability to save constraint equations
with metric files. These are auxiliary equations which must be satisfied
by functions contained in the metric and can later be applied to objects
calculated from the metric or basis.

For example, in Kruskal's coordinates for the Schwarzschild spacetime,
\[
  ds^2 = 16 \frac{m^2(r-2m)}{r(u^2-v^2)}(du^2 - dv^2) + r^2 d\Omega^2,
\]
the function $r(u,v)$ is required to satisfy the differential relations
\[
  \frac{\partial r}{\partial u} = 4 \frac{m u (r-2m)}{r(u^2 - v^2)}, 
    \qquad
  \frac{\partial r}{\partial v} = -4\frac{m v (r-2m)}{r(u^2 - v^2)}.
\]

Constraint equations can be included when the metric is created using
\grcmd{makeg}. Alternatively, the command \grcmd{grconstraint} allows you
to add or modify the constraints equations associated with a given metric.
It takes the following form:\\
%
\begin{cmdspec}
  \label{spec:grconstraint}
  \grcmdline{grconstraint ( \grarg{metricName} )}

  \begin{description}
    \item[\grarg{metricName}] -- the name of a previously loaded metric.
  \end{description}

  \grexample{grconstraint ( schw )}
\end{cmdspec}

Constraint equations may be added to a metric, removed or re-arranged 
using this command. \grcmd{grconstraint} is menu driven and prompts for 
addition/modification/deletion of constraint equations.

The constraints are not invoked
automatically during calculation of tensors, but must be applied explicitly
using \grcmd{gralter} or \grcmd{grcalcalter} (see Booklet \grCalcRef). A
command sequence which would calculate the Ricci tensor for a metric and
then apply the metric constraint equations to the result is:\\

\begin{ttfamily}
  \noindent > grcalc ( R(dn,dn) ):\\
  > gralter ( R(dn,dn), consr ):\\
\end{ttfamily}

Note that constraint equations modified using \grcmd{grconstraint} are
not saved automatically in the metric file. The command \grcmd{grsaveg}
(see Section \ref{sec:grsaveg})
should be used if the modified constraints are to be associated with the
metric in future GRTensorIII sessions. They will then be loaded automatically
when the metric is loaded using \grcmd{qload} or \grcmd{grload}.
%
%------------------------------------------------------------------------------
\section{Creating a null tetrad from a metric}\label{sec:nptetrad}
%------------------------------------------------------------------------------
%
The command \grcmd{nptetrad} can be used to create a null tetrad
(whose basis vectors satisfy the inner product given by the
$\eta_{(a)(b)}$ of Eq. (\ref{eq:nullip})) from a 4-dimensional
metric, $g_{ab}$.  The format of the command is:\\
%
\begin{cmdspec}
  \label{spec:nptetrad}
  \grcmdline{nptetrad ( [\grarg{lnSpace}] )}

  \begin{description}
    \item[\grarg{lnSpace}] -- (\textit{optional}) a pair of coordinates
      which describe a preferred timelike 2-space in which the $l$ and $n$
      vectors are to reside.
  \end{description}

  \grexample{nptetrad ( [t,r] )}
\end{cmdspec}

The command finds a set of null 1-forms satisfying the NP inner
product, Eq. (\ref{eq:nullip}), for a given metric, $g_{ab}$.
The forms are saved as the object \texttt{e(bdn,dn)}.\\

The algorithm for constructing the tetrad proceeds as follows. As a first
step, a set of vectors corresponding to the columns of the covariant
metric are used as a basis,
\[
  e_{(1)} := g_{1a}, \qquad e_{(2)} := g_{2a}, \qquad  e_{(3)} := g_{3a},
  \qquad e_{(4)} := g_{4a}.
\]
If a pair of coordinates are listed as the optional \grarg{lnSpace}
argument, then the corresponding vectors are used to construct the $l$
and $n$ vectors. For instance, if the coordinates of the spacetime are
given as $(t,r,\theta,\phi)$ and the argument is given as
\texttt{[t,r]}, then the $l$ and $n$ vectors will be constructed as
linear combinations of the vectors
\begin{align*}
  e_{(1)} & := [g_{t1}, g_{t2}, g_{t3}, g_{t4} ], \\
  e_{(2)} & := [g_{r1}, g_{r2}, g_{r3}, g_{r4} ].
\end{align*}
The remaining pair are used to construct $m$ and $\bar{m}$. 

If the argument is omitted, a valid null tetrad will still be created,
though the construction might be somewhat less efficient if the
computer is left to find appropriate vectors on its own.\footnote{In
previous versions of GRTensorIII, the \grcmd{nptetrad} command required
the specification of a timelike vector as input. The new algorithm
makes better use of the coordinates of the spacetime (especially null
coordinates) in constructing the tetrad. Note that in versions of
GRTensorIII previous to 1.50, the output of \grcmd{nptetrad} was in the
form of the contravariant vectors
\texttt{e(bdn,up)}$=e_{(a)}{}^b$. The natural form of output for the
new algorithm is the covariant 1-forms, and thus the revised
\grcmd{nptetrad} has as its output the components of
\texttt{e(bdn,dn)} $= e_{(a)b}$.}  It will first attempt to locate
null vectors among the columns of the metric. If none exist, then it
picks a pairs of non-null column vectors and attempts to carry out an
orthonormalization procedure to construct a tetrad satifying the NP
inner product. If the first attempt fails (usually because it has
chosen two spacelike vectors to construct $l$ and $n$), it cycles
through pairs of vectors until it is successful, or all unique
combinations of columns of the metric have been exhausted, in which
case a warning is issued.\\

A problem arises because of the need to take square roots in the
normalization process. Because of the unpredictable usage of the
imaginary number $i=\sqrt{-1}$ by the MapleV \grcmd{sqrt} command
acting on symbolic expressions, it is not always possible for the
algorithm to determine when a null tetrad is of the NP form or
not. Occasionally a tetrad can be constructed which satisfies the NP
inner product, but for which $l$ and $n$ contain imaginary components,
or for which the vector $\bar{m}$ is not actually the complex
conjugate of $m$.  Since the algorithm can not reliably test these
criteria, it is the user's responsibility to check that the objects
\texttt{e(bdn,dn)} conform to these properties of a true NP
tetrad.\footnote{It is the authors' experience that a true NP tetrad
is always produced when the \grarg{lnSpace} argument is specified,
or when the spacetime contains at least one null coordinate.}\\

The appearance of large terms in the components of the basis,
especially terms containing radicals, is a common difficulty with
tetrads produced by the \grcmd{nptetrad} command. To ensure that
future calculations are optimized, it is important to express the
basis in as simple a form as possible. To do this, apply the relevant
\grcmd{gralter} commands, especially \texttt{radical} and
\texttt{radsimp} to the basis vectors, \texttt{e(bdn,up)} until they
seem to be fully simplified. (Simplification using \grcmd{gralter} is
described more fully in Booklet \grCalcRef).\\

Note that the implementation of the Newman-Penrose formalism in
GRTensorIII follows that of the original specification of
\cite{newman/penrose:1962}. As such, it requires that the spacetime
has a -2 signature. This conflicts with the Landau-Lifshitz spacelike
convention for which GRTensorIII metrics are generally defined. If
neccessary, the \grcmd{nptetrad} command will prompt the user, asking
if the signature of the spacetime is to be reversed when constructing
the tetrad. An affirmative response will cause the sign of the
components of $g_{ab}$ and $g^{ab}$ to be reversed.\footnote{Only the
signatures of these objects are reversed. Objects previously calculated
from the metric will not be updated to the new spacetime
signature. For this reason, it is safest to run \grcmd{nptetrad} near
the beginning of a session before further calculations with the metric
are carried out in order to avoid sign conflicts.} If the spacetime
signature is anything other than +2 or -2, the \grcmd{nptetrad}
command can not be used.\\

Tetrads created using \grcmd{nptetrad} can be saved using \grcmd{grsaveg}
For more information regarding GRTensorIII calculations in a null
tetrad, see Booklet \grBasisRef.  The components of the basis created
by \grcmd{nptetrad} can be saved using the \grcmd{grsaveg} command
(Section \ref{sec:grsaveg}).
%
%------------------------------------------------------------------------------
\subsection{Tetrad rotations}
%------------------------------------------------------------------------------
%
An alternative method of simplifying null tetrads is to perform a rotation
of the basis vectors. Such rotations can be divided into three 
classes \cite{newman/penrose:1962}:
\begin{description}
  \item[Class I:] -- leaves basis vector $l$ unchanged. This rotation is
    specified by a single complex-valued parameter, $a$. The basis vectors
    transform according to:
    \begin{align*}
      l & \longrightarrow l, \\
      n & \longrightarrow n + a^\ast m + a \bar{m} + a a^\ast l, \\
      m & \longrightarrow m + a l, \\
      \bar{m} & \longrightarrow \bar{m} + a^\ast l.
    \end{align*}
%
  \item[Class II:] -- leaves basis vector $n$ unchanged. This rotation is
    specified by a single complex-valued parameter, $b$. The basis vectors
    transform according to:
    \begin{align*}
      l & \longrightarrow l + b^\ast m + b \bar{m} + b b^\ast n,\\
      n & \longrightarrow n, \\
      m & \longrightarrow m + b n, \\
      \bar{m} & \longrightarrow \bar{m} + b^\ast n.
    \end{align*}
%
  \item[Class III:] -- This rotation is specified by two real-valued 
    parameters: $\theta$ determines a rotation in the $(m,\bar{m})$ plane,
    and $A$ determines a boost in the $n$ direction. The basis vectors
    transform according to:
    \begin{align*}
      l & \longrightarrow A^{-1} l, \\
      n & \longrightarrow A n, \\
      m & \longrightarrow \exp (i\theta) m, \\
      \bar{m} & \longrightarrow \exp (-i\theta) \bar{m}.
    \end{align*}
  \end{description}

The \grcmd{nprotate} command takes the rotation class and its corresponding
parameters as arguments, and performs the specified rotation on the basis
vectors \texttt{e(bdn,up)}. The command has the form:\\
%
\begin{cmdspec}
  \label{spec:nprotate}
  \grcmdline{nprotate ( \grarg{class}, \grarg{parm1}, \grarg{parm2} )}

  \begin{description} 
    \item[\grarg{class}] -- the rotation class, as defined above. This
      argument takes the value \texttt{I}, \texttt{II}, or \texttt{III},
      depending on the desired rotation.

    \item[\grarg{parm1}] -- the first rotation parameter,
      $\text{Re}(a)$ for Class I rotations, $\text{Re}(b)$ for Class
      II rotations, or $A$ for Class III rotations.

    \item[\grarg{parm2}] -- the second rotation parameter,
      $\text{Im}(a)$ for Class I rotations, $\text{Im}(b)$ for Class
      II rotations, or $\theta$ for Class III rotations.
  \end{description}

  \grexample{nprotate ( III, sqrt(2)*sqrt(1-2*m/r), 0 )}
\end{cmdspec}

Note, in the current versions of GRTensorIII, only rotations of the basis
can be performed. The action of basis rotations on other curvature tensors
has not yet been implemented. Thus, curvature tensors must be recalculated
from the rotated basis.\\

The newly created tetrad is initialized as the default for subsequent
calculations (see Section \ref{sec:default}).
The components of the rotated basis are not saved automatically, but can be
saved for future used with the \grcmd{grsaveg} command (Section 
\ref{sec:grsaveg}).
%
%------------------------------------------------------------------------------
\section{Metrics from an array} \label{sec:grnewmetric}
%------------------------------------------------------------------------------
%
The \grcmd{grnewmetric} command defines a new metric by copying the components
of a covariant two-index tensor to a new metric tensor. The format of the 
command is:\\
%
\begin{cmdspec}
  \label{spec:grnewmetric}
  \grcmdline{grnewmetric ( \grarg{newMetric}, \grarg{objectName},
    [\grarg{coords}] )}

  \begin{description}
    \item[\grarg{newMetric}] -- the name to be assigned to the newly created
      metric.
    \item[\grarg{oldObject}] -- the name of a two-index covariant GRTensor
      object.
    \item[\grarg{coords}] -- \textit{(optional)} a list of coordinate names
      for the new metric, if they are different from those of the object from
      which it is created.
  \end{description}

  \grexample{grnewmetric ( confRW, confg(dn,dn) )}
\end{cmdspec}

The newly created metric becomes the default metric for the current
session. It is not saved automatically to a metric file, but can be saved
using the \grcmd{grsaveg} command.
%
%------------------------------------------------------------------------------
\section{Saving modified spacetimes} \label{sec:grsaveg}
%------------------------------------------------------------------------------
%
The \grcmd{grsaveg} command is used to save the information associated with
a given metric or basis. The format of the command is:\\
%
\begin{cmdspec}
  \label{spec:grsaveg}
  \grcmdline{grsaveg ( \grarg{saveName}, [\grarg{metricPath}] )}

  \begin{description}
    \item[\grarg{saveName}] -- the name (without the `\texttt{.mpl}' extension)
      of the file in which the metric (or basis) information is to be stored.
    \item[\grarg{metricPath}] -- (\textit{optional}) the directory in which
      the the metric file is to be placed. If this argument is not specified
      the directory stored in the global option variable
      \texttt{grOptionMetricPath} is used instead.
  \end{description}

  \grexample{grsaveg ( newSchw )}
\end{cmdspec}

The current default spacetime is saved to the file specified by
\grarg{saveName}\texttt{.mpl} in the directory specified by the global
\texttt{grOptionMetricPath} variable or the \grarg{metricPath}
argument, if it is specified.  The information saved includes:
\begin{center}
  \begin{tabular}{ll}
    coordinates, & \texttt{x(up)}, \\
    metric components, & \texttt{g(dn,dn)} \textit{(if they are assigned for
      the default spacetime)},\\
    basis components, & \texttt{e(bdn,dn)} and/or \texttt{e(bdn,up)}
      \textit{(if assigned)},\\
    signature, & \texttt{sig} \textit{(if assigned)}, \\
    constraint equations, & \texttt{constraint} \textit{(if assigned)}, \\
    text description, &  \texttt{Info} \textit{(if assigned)}.\\
  \end{tabular}
\end{center}
(See Booklet \grCalcRef~or the \texttt{?grt\_objects} and
\texttt{?grt\_basis} online help pages for descriptions of these
objects.)\\

If both metric components (\texttt{g(dn,dn)}) and basis components
(\texttt{e(bdn,dn)} and/or \texttt{e(bdn,up)}) have been assigned for
the current default spacetime, then the user is prompted as to which
types of information are to be saved.\\

Note that MapleV \textit{can not check} for the existence of files before
it performs the \texttt{write} operation. Thus if a metric file with the
specified name already exists in the \texttt{grOptionMetricPath} directory,
then \textit{it will be overwritten}.
%
%------------------------------------------------------------------------------
\subsection{Metric files} \label{sec:metricfiles}
%------------------------------------------------------------------------------
%
Metric files are simply ASCII files containing the coordinate
components of the metric or basis. They can be modified (or created)
with a text editor. By default, files are saved in
the directory specified by the \texttt{grOptionMetricPath} variable
with a file name of the form \grarg{metricName}\texttt{.mpl}.\\

\begin{figure}
  \begin{ttfamily}
    \begin{tabbing}
      Ndim\_ := 4:\\

      x1\_ := r:\\
      x2\_ := theta:\\
      x3\_ := phi:\\
      x4\_ := t:\\
      sig\_ := 2:\\
      g11\_ := diff ( R(r,t),r)\^{}2/(1+f(r) ):\\
      g22\_ := R(r,t)\^{}2:\\
      g33\_ := R(r,t)\^{}2*sin(theta)\^{}2:\\
      g44\_ := -1:\\

      constraint\_ :=\= [ \= diff(diff(R(r,t),r),t) = 
        (2*diff(m(r),r)/R(r,t) \\
        \> \> - 2*m(r)*diff(R(r,t),r)/R(r,t)\^{}2 \\
        \> \> + diff(f(r),r))/(2*sqrt(2*m(r)/R(r,t)+f(r))),\\ 
        \> \> diff(R(r,t),t) = sqrt(2*m(r)/R(r,t)+f(r)),\\
        \> \> diff(diff(R(r,t),t),t) = -m(r)/R(r,t)\^{}2,\\
        \> \> diff(diff(diff(R(r,t),t),r),t) = 
          -diff(m(r),r)/R(r,t)\^{}2 + \\
        \> \> 2*m(r)*diff(R(r,t),r)/R(r,t)\^{}3\\
        \> ]:\\

      Info\_:=
        `The Tolman dust solution (Proc. Nat. Acad. Sci. 20, 169,1934)`:
    \end{tabbing}
  \end{ttfamily}
  \caption{The metric file \texttt{dust1.mpl} from the standard metric
    library.}
  \label{fig:metricfile}
\end{figure}
Every metric or basis description file must contain a line 
`\texttt{Ndim\_ := $n$:}'
where \texttt{$n$} is a number giving the dimension of the spacetime.\\

Every metric or basis file must contain a set of assignments to the
variables \texttt{x1\_}, \texttt{x2\_}, $\ldots$,
\texttt{x\textsl{n}\_}, giving the names of the coordinates of the
spacetime.\\

If the file describes a covariant metric, $g_{ab}$, it must contain a
set of assignments to the variables \texttt{g11\_}, \texttt{g12\_},
$\ldots$, \texttt{g\textsl{nn}\_}, giving the components of the
metric. Only the upper diagonal of the metric needs to be
specified. Any element of the upper diagonal that is not assigned is
assumed to have the value zero.\\

If the file describes a set of basis vectors, it must contain
assignments to the variables \texttt{b11\_}, \texttt{b12\_}, $\ldots$,
\texttt{b\textsl{nn}\_}, giving the components of contravariant basis
vectors, or assignments to \texttt{bd11\_}, \texttt{bd12\_}, $\ldots$,
\texttt{bd\textsl{nn}\_}, in the case of covariant basis vectors. The
inner product must also be specified by assignments to the variables
\texttt{eta11\_}, \texttt{eta12\_}, \ldots, \texttt{eta\textsl{nn}\_}.
Variables which are not assigned are assumed to have value zero when
the basis is loaded.\\

If the signature of the spacetime has been assigned, it is assigned
to the variable \texttt{sig\_} in the metric file. This variable
is of the form of an integer $|s|\leq n$.\\

If the metric or basis possesses constraint equations (see above),
these are represented as a MapleV list assigned to the variable
\texttt{constraint\_}.  Text descriptions can be included by assigning
a string to the name \texttt{Info\_}.\\

An example of a metric file which specifies a spacetime using a covariant
metric is given in Fig.~(\ref{fig:metricfile}). Many examples can
be found in the metrics directory which is provided with the GRTensorII
installation, as well as from the GRTensorIII world-wide-web pages
\cite{www}.
%
%------------------------------------------------------------------------------
\vfill
\bibliographystyle{unsrt}
\bibliography{grtensor}
%------------------------------------------------------------------------------
\pagebreak
\section*{Commands described in this booklet:}
  \noindent
  \grcmdline{makeg ( \grarg{metricName}, [\grarg{metricPath}] )}
    \dotfill \pageref{spec:makeg}\\

  \noindent
  \grcmdline{qload ( \grarg{metricName} )}
    \dotfill \pageref{spec:qload}\\

  \noindent
  \grcmdline{grload ( \grarg{metricName}, \grarg{metricFile} )}
    \dotfill \pageref{spec:grload}\\

  \noindent
  \grcmdline{grmetric ( \grarg{metricName} )}
    \dotfill \pageref{spec:grmetric}\\

  \noindent
  \grcmdline{grtransform ( \grarg{oldmetric}, \grarg{newmetric},
    \grarg{xform} )}
    \dotfill \pageref{spec:grtransform}\\

  \noindent
  \grcmdline{grconstraint ( \grarg{metricName} )}
    \dotfill \pageref{spec:grconstraint}\\

  \noindent
  \grcmdline{nptetrad ( \grarg{lnSpace} )}
    \dotfill \pageref{spec:nptetrad}\\

  \noindent
  \grcmdline{nprotate ( \grarg{class}, \grarg{parm1}, \grarg{parm2} )}
    \dotfill \pageref{spec:nprotate}\\

  \noindent
  \grcmdline{grnewmetric ( \grarg{newMetric}, \grarg{objectName},
    [\grarg{coords}] )}
    \dotfill \pageref{spec:grnewmetric}\\

  \noindent
  \grcmdline{grsaveg ( \grarg{saveName}, [\grarg{metricPath}] )}
    \dotfill \pageref{spec:grsaveg}\\
%------------------------------------------------------------------------------
\vfill
\large
\noindent The information contained in this booklet is also available from the
following online help pages:\\

\noindent\texttt{?grt\_metrics}, \texttt{?grt\_objects}, \texttt{?grt\_basis},
\texttt{?groptions}, \texttt{?makeg}, \texttt{?qload},
\texttt{?grload}, \texttt{?grmetric}, \texttt{?grtransform},
\texttt{?grconstraint}, \texttt{?nptetrad}, \texttt{?nprotate},
\texttt{?grnewmetric}, \texttt{?grsaveg}.\\
%
%------------------------------------------------------------------------------
\end{document}
%==============================================================================
%
