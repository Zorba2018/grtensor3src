%=================================================================-*-LaTeX-*-==
% GRTensorIII 1.50 Manual
% Booklet A: Introduction and overview
%
% Denis Pollney
% July 1996
%==============================================================================
\documentclass{article}
\usepackage{amsmath}
% Maple package causes a failure of the title text to rotate
% can just return through the missing maplelatex stmts while building
% and document appears to be ok
%\usepackage{maplestd2e}
\usepackage{grtensor}
\usepackage{grbooks}
\usepackage{hyperref}
%------------------------------------------------------------------------------
\begin{document}
%\setlength{\footrulewidth}{\headrulewidth}
\grtitle{\grIntroTitle}
\grlabel{\grIntroLabel}
\grdate{Nov 2016}
\grtitlepage
%
%==============================================================================
% Body.
%==============================================================================
\copyrightpage
\noindent GRTensorIII is a package for the calculation and manipulation of
components of tensors and related objects. Rather than focus upon a
specific type or method of calculation, the program has been designed
to operate efficiently for a wide range of applications and allows the
use of a number of different mathematical formalisms.  Algorithms are
optimized for the individual formalisms and transformations between
formalisms has been made simple and intuitive. Additionally, the
package allows for customization and expansion with the ability to
define new objects, user-defined algorithms, and add-on libraries.

Regardless of the algorithm or formalism used, it is often the case
that only certain simplifications applied at a crucial stages can make
some problems tractable. For this reason, GRTensorIII has been designed
to provide full control of the calculation path and of the
simplifications to be performed at each stage. Once a calculation is
completed, a variety of commands are available for the manipulation
and simplification of results.

In designing the package, emphasis has also been placed on the
interface, allowing simple user input, as well as presenting readable
output. Metrics and basis vectors are easily defined and can be saved
for later use.  Calculations are specified in an intuitive manner
using a minimal number of commands.\\

This booklet introduces a number of characteristics and features of
the GRTensorIII computer algebra package, emphasizing its input and
output facilities, the ability to define new tensor objects, and the
ability to use and switch between a number of different formalisms
(classical tensor methods, general bases, and null tetrads). This
booklet itself is intended only as a brief overview. More detailed
information regarding the commands introduced here can be found
in Booklets \grMakegLabel--\grSetupLabel:
\begin{center}
  \begin{tabular}{rl}
    \textsl{Booklet}
    & \grMakegRef\\
    & \grCalcRef\\
    & \grDefRef\\
    & \grBasisRef\\
    & \grHyperRef \\
  \end{tabular}
\end{center}
Together these articles are intended to serve as both a user's guide
and command reference for the GRTensorIII package. Unfortunately
because of this dual purpose, there are some shortfalls in their
ability to perform each task: Someone looking for tutorial information
might find themselves buried in uninteresting technical points,
while experienced users will find some sections overly pedagogic.

It is not intended that the booklets be read linearly from start to
finish. To begin, only the information from the first sections of the
\textsl{\grMakegTitle} and \textsl{\grCalcTitle} booklets are needed
to perform some sophisticated calculations. Once familiarity with the
interface is gained, the ability to define new tensors and build
customized libraries, described in Booklet \grDefRef, greatly increase
the power of the package.  For the first time user, probably the best
way to approach the package is to study and execute some of the
demonstrations available in the worksheets directory provided with 
the package, referring to these booklets and the online help as
necessary.\\

Major Releases: \\

\textbf{Release 2.00:} Add junction and hypersurface capability. \\
 
\textbf{Release 1.00:} GRTensorIII is a adaptation of GRTensorII
version 1.50 developed in 1994-1999. It is primarily a refactoring
of internal data representation with some updates to the ''modern''
Maple environment. \\
\begin{itemize}
  \item move all variables inside a scoped Maple package
  \item include all extension objects from GRII (basic etc.) directly
  in the main package
  \item support Maple worksheet dialog based input
  \item remove some seldom used commands (grloaddef, grsavedef, 
  grloadobj, grsaveobj)
  \item no changes have been made to the expressions used to define
  the standard tensor objects\\
\end{itemize}

The version of GRTensorIII described in these booklets runs within the
Maple environment. It has been tested with Maple 2016 in command line and
worksheet mode. 
We encourage users to check back periodically for new
versions of software, documentation, and demonstrations, at \href{url}{http:/github.com/grtensor/grtensor}.
This represents the "official" version. \\

\getgrt \\

%
%------------------------------------------------------------------------------
\section{Startup}
%------------------------------------------------------------------------------
The GRTensorIII package is distributed as Maple package. 
To start the GRTensorIII package, in a new Maple session ensure
that the location of \text\tt{grtensor/lib} is in the Maple path. It can be added
by:\\

\noindent\texttt{libname := libname, "path";}

The package is loaded in the standard manner:\\

\noindent\texttt{with(grtensor);}\\

Build-in help pages can be viewed via:\\

\noindent\texttt{?grtensor}\\

The examples described below are designed to be run in a single
GRTensorIII session.
%------------------------------------------------------------------------------
\section{Calculating components}
%------------------------------------------------------------------------------
Tensors are specified by their name and index configuration, using the
abbreviations {\tt dn} and {\tt up} to indicate covariant and
contravariant indices, respectively. For instance, the Ricci tensor
could be referenced
\[
  \text{\tt R(dn,dn)}\quad (= R_{ab}),\qquad
  \text{\tt R(up,dn)}\quad (= R^a_{\ b}), \qquad
  \text{\tt R(bup,bdn)}\quad (= R^{(a)}_{\ \ (b)}),
\]
where the last example uses the labels {\tt bup} and {\tt bdn} to
reference the components of the Ricci tensor in terms of a basis
rather than metric coordinates.

Three commands are used most frequently within a GRTensorIII session:
\begin{description}
  \item[grcalc] calculates the components of tensors;
  \item[grdisplay] displays the components;
  \item[gralter] applies simplification routines to tensors;
\end{description}
Thus, to calculate the covariant Ricci tensor for the predefined Kerr metric,
we might use the command sequence\\

\noindent{\texttt{> qload ( kerr ):\\
> grcalc ( R(dn,dn) ):\\
\indent Calculating g(dn,dn,pdn) for kerr ... Done. \\
\indent Calculating Chr(dn,dn,dn) for kerr ... Done. \\
\indent Calculating detg for kerr ... Done. \\
\indent Calculating g(up,up) for kerr ... Done. \\
\indent Calculating Chr(dn,dn,up) for kerr ... Done. \\
\indent Calculating R(dn,dn) for kerr ... Done. \\[\baselineskip]
%
> grdisplay ( R(dn,dn) ):}
\begin{center}
  \textit{For the kerr spacetime:}\\
  \textit{Covariant Ricci}
\end{center}
\begin{eqnarray*}
\lefteqn{{{\it R\:}_{{r}}}\,{{}_{{r}}}= - {a}^{2} \left( {\vrule 
height0.44em width0em depth0.44em} \right. \! \! {r}^{2} - {r}^{2
}\,{\rm sin}(\,{ \theta}\,)^{2} + 2\,{r}^{2}\,{\rm cos}(\,{ 
\theta}\,)^{4} - 3\,{r}^{2}\,{\rm cos}(\,{ \theta}\,)^{2} + 2\,{r
}^{2}\,{\rm sin}(\,{ \theta}\,)^{2}\,{\rm cos}(\,{ \theta}\,)^{2}
 + {a}^{2}\,{\rm cos}(\,{ \theta}\,)^{4}} \\
 & & \mbox{} - {a}^{2}\,{\rm cos}(\,{ \theta}\,)^{2} + {a}^{2}\,
{\rm cos}(\,{ \theta}\,)^{2}\,{\rm sin}(\,{ \theta}\,)^{2} \! 
\! \left. {\vrule height0.44em width0em depth0.44em} \right) 
 \left/ {\vrule height0.44em width0em depth0.44em} \right. \! \! 
 \left( \! \, \left( \! \, - 1 + {\rm cos}(\,{ \theta}\,)^{2}\,
 \!  \right) \,(\,{r}^{2} - 2\,{m}\,{r} + {a}^{2}\,)\, \left( \! 
\,{r}^{2} + {a}^{2}\,{\rm cos}(\,{ \theta}\,)^{2}\, \!  \right) 
^{2}\, \!  \right)\\
\lefteqn{R_{\theta\theta} = \ldots \text{etc.}} 
\end{eqnarray*}
At this point, since no simplifications of the results have 
been applied, the components of $R_{ab}$ are not obviously zero
as we would expect. The following commands are needed:\\

\noindent\texttt{> gralter ( R(dn,dn), trig ):\\
> grdisplay ( R(dn,dn) ):}\\
\begin{center}
  \textit{For the kerr spacetime:\\
   Covariant Ricci\\
   R(dn,dn) = All components are zero}
\end{center}
The first command applies trigonometric simplification to the
components of \texttt{R(dn,dn)}. Upon using \grcmd{grdisplay}, we see
that the the simplification has indeed reduced the components to zero.

Note that for a more complicated metric, it may be the case that it is
better to carry out the computation in stages. That is, we can first
calculate the Christoffel symbols and apply simplifications, then the
Ricci tensor, and finally, for instance, the Ricci scalar.

A number of alternate simplification routines exist within
\grcmd{gralter}, including trigonometric, radical, expansion and
factorization techniques.  The command \grcmd{grmap} can also be used
to apply \textit{any} Maple function to the components of a tensor.

The commands listed in this Section are described in more detail in
Booklet \grCalcRef. The calculation of objects in a coordinate basis
is described in Booklet \grBasisRef.
%
%------------------------------------------------------------------------------
\section{Input of spacetimes}
%------------------------------------------------------------------------------
Spacetimes are specified within GRTensorIII through use of the
\grcmd{makeg} facility. They can be entered in the following forms:
\begin{description}
  \item[Metric, $g_{ab}$] A $n\times n$ symmetric matrix;
  \item[Line element] An equation of the standard form $ds^2 = \ldots$;
  \item[Non-holonomic basis] A set of basis vectors (covariant or
    contravariant) with an inner product;
  \item[Null tetrad] As described by the Newman-Penrose formalism.
\end{description}
As an illustrative (but computationally trivial) example, we consider
the twisting Einstein-Maxwell solution described on p. 135 of Kramer,
et al.  \cite{krameretal},
\begin{align*}
  kds^2 & = -2\left( x^a du - \frac{dy}{(a+1)x}\right)
    \left[ dt + a(t dx + dy)/x + f(t)\left( x^a du - \frac{dy}{(a+1)x}
    \right)\right] \nonumber\\
    & \qquad + (dx^2 + dy^2)(t^2+1)/2x^2,
\end{align*}
We can enter this metric directly in this form using \grcmd{makeg}, giving
it the name `\texttt{sm}':\\

\indent\em{In Maple worksheets the following prompts will appear
in dialog boxes. In the example here we show the responses in 
command-line mode.}\\

\noindent\texttt{> makeg ( sm ):\\
\indent\textsl{\small Makeg 2.0: GRTensorIII metric/basis entry utility\\
\indent To quit makeg, type 'exit' at any prompt.}}
\begin{tabbing}
\indent\texttt{\textsl{Do you wish to enter a }}
  \= \texttt{\textsl{1) metric [g(dn,dn)],}}\\
  \> \texttt{\textsl{2) line element [ds],}}\\
  \> \texttt{\textsl{3) non-holonomic basis [e1...e4], or}}\\
  \> \texttt{\textsl{4) null tetrad [l,n,m,mbar]?}}
\end{tabbing}
\texttt{> 2:\\
\indent{\textsl{\small Enter coordinates as a LIST
  (e.g. [r,theta,phi,t]):}}\\
> [ t, x, y, u ]:}\\

\texttt{\textsl{{\small Enter the line element using 'd[coord]' to indicate
  differentials.\\
\indent (for example,  r\^{}2*(d[theta]\^{}2 + sin(theta)\^{}2*d[phi]\^{}2)\\
\indent $[$Type 'exit' to quit makeg$]$\\
\indent  ds\^{}2 = }}}\\

\noindent\texttt{> 1/k*(-2*(x\^{}a*d[u] - d[y]/((a+1)*x))*(d[t]
  + a*(t*d[x] \\
\indent + d[y])/x + f(t)*(x\^{}a*d[u]-d[y]/((a+1)*x))) + (d[x]\^{}2 +\\
\indent d[y]\^{}2)*(t\^{}2+1)/(2*x\^{}2)):\\
}

\noindent Note that in the above expression the coordinate differentials
are labeled using the variables {\tt d[}{\em coordinate}{\tt ]},
allowing the line element to be entered in exactly the form that it is
given in \cite{krameretal}. The output from {\tt makeg} is:
\begin{align*}
  ds^2 & = 2\frac{dt\ dy}{k(a+1)x} - 2\frac{x^a dt\ du}{k}
    + \left(\frac{1}{2}\frac{t^2}{kx^2} + 
    \frac{1}{2}\frac{1}{kx^2}\right) dx^2
    + 2\frac{a\ t\ dx\ dy}{k(a+1)x^2}\\
  & - 2\frac{x^a a\ t\ dx\ du}{kx} + \left( 2\frac{a}{k(a+1)x^2}
    - 2\frac{f(t)}{k(a+1)^2 x^2} + \frac{1}{2}\frac{t^2}{kx^2} +
    \frac{1}{2}\frac{1}{kx^2}\right)dy^2\\
  & + 2\left(-\frac{x^a a}{kx} + 2\frac{x^a f(t)}{k(a+1)x}\right)dy\ du
    - 2\frac{(x^a)^2 f(t) du^2}{k}
\end{align*}
\indent\texttt{\textsl{\small makeg completed.}}\\
The parameter values can now be set for the particular Type-III case
studied by Siklos and MacCallum \cite{maccallum/siklos:1981} (referred
to in \cite{krameretal} on p. 135):\\

\noindent\texttt{> a := 1/2: k := -32/(78*lambda):
  f(t) := (13*t\^{}2+17)/32:}\\

\noindent For this metric, we calculate the
(rather uninteresting, in this case) components of the Ricci tensor:\\

\noindent\texttt{> grcalc ( R(dn,dn) ):}\\
\noindent\texttt{> grdisplay ( \_ ):}
\begin{center}{\em For the sm metric:}\end{center}
\[
  R_{ab} = \left[
    \begin{array}{cccc}
      0 & 0 & -\frac{13}{8}\frac{1}{x} & \frac{39}{16}\sqrt{x} \\
      0 & - \frac{39}{32}\frac{t^2+1}{x^2} & -\frac{13}{16}\frac{t}{x^2} &
        \frac{39}{32}\frac{t}{\sqrt{x}} \\
        -\frac{13}{8}\frac{1}{x} & -\frac{13}{16}\frac{t}{x^2} &
      -\frac{65}{192}\frac{5+t^2}{x^2} & 
        -\frac{13}{128}\frac{5+13t^2}{\sqrt{x}} \\
      \frac{39}{16}\sqrt{x} & \frac{39}{32}\frac{t}{\sqrt{x}} &
        -\frac{13}{128}\frac{5+13t^2}{\sqrt{x}} &
        \frac{39}{256}x(13t^2+17)
    \end{array}
    \right]
\]

The specification of metrics and bases is described more fully in Booklet
\grMakegRef. The \grcmd{grcalc}, \grcmd{gralter} and \grcmd{grdisplay}
commands are the subject of Booklet \grCalcRef.
%
%------------------------------------------------------------------------------
\section{The Newman-Penrose formalism}
%------------------------------------------------------------------------------
A more complicated problem is that of determining the Petrov type of
this metric. In order to do this in GRTensorIII, the metric must be
cast in the form of a Newman-Penrose (NP) null tetrad. Since no tetrad
obviously presents itself from the form of the metric given above, we
can use the command \grcmd{nptetrad} to generate one. This command
constructs a set of null basis vectors satisfying the properties of
a basis for the Newman-Penrose formalism.\footnote{Since the -2 signature
convention of the NP formalism conflicts with the +2 Landau-Lifschitz
convention recommended for metrics in GRTensorIII, the construction
of the tetrad also involves a signature change. This is carried out
automatically if requested.}  Naturally, this tetrad
may not be optimal for further calculation, but is often a good start
to which further simplifications can be applied.\\

\noindent\texttt{> nptetrad([u,t]):}\\
\indent\texttt{The metric signature of the sm spacetime is +2.}\\
\indent\texttt{In order to create an NP-tetrad, the signature of g(dn,dn)
 will be changed}\\
\indent\texttt{to -2.}\\
\indent\texttt{Continue? (1=yes):}\\
\noindent\texttt{> 1:}\\
\indent\texttt{The signature of the sm spacetime is now -2.\\}

\begin{maplelatex}
\[
{\it Null\:tetrad\:(covariant\:components)}
\]
\end{maplelatex}
\begin{maplelatex}
\[
{{l}_{{a}}}= \left[
{\begin{array}{cccc}
 - \,{\displaystyle \frac {9}{4}}\,{ \lambda}\,\sqrt {{x}} & 
 - \,{\displaystyle \frac {9}{8}}\,{\displaystyle \frac {{ 
\lambda}\,{t}}{\sqrt {{x}}}} & {\displaystyle \frac {3}{64}}\,
{\displaystyle \frac {{ \lambda}\,(\, - 7 + 13\,{t}^{2}\,)}{
\sqrt {{x}}}} &  - \,{\displaystyle \frac {9}{128}}\,{ \lambda}
\,{x}\,(13{t}^{2}+17)
\end{array}}
\right] 
\]
\end{maplelatex}
\begin{maplelatex}
\[
{{n}_{{a}}}= \left[
{\begin{array}{rrcr}
0 & 0 &  - \,{\displaystyle \frac {2}{3}}\,{\displaystyle \frac {
1}{{x}^{3/2}}} & 1
\end{array}}
\right] 
\]
\end{maplelatex}
\begin{maplelatex}
\[
{{m}_{{a}}}= \left[
{\begin{array}{rrcr}
0 &
{\frac {3/4\,i \left( {t}^{2}+1 \right) \lambda}{x\sqrt {- \left( {t}^
{2}+1 \right) \lambda}}}
3/4\,{\frac { \left( {t}^{2}+1 \right) \lambda}{x\sqrt {- \left( {t}^{
2}+1 \right) \lambda}}} & 0
\end{array}}
\right] 
\]
\end{maplelatex}
\begin{maplelatex}
\[
{{\it mbar}_{{a}}}= \left[
{\begin{array}{rccr}
0 & {\frac {-3/4\,i \left( {t}^{2}+1 \right) \lambda}{x\sqrt {- \left( {t}
^{2}+1 \right) \lambda}}}
3/4\,{\frac { \left( {t}^{2}+1 \right) \lambda}{x\sqrt {- \left( {t}^{
2}+1 \right) \lambda}}}
 & 0
\end{array}}
\right] 
\]
\end{maplelatex}

\indent\texttt{The null tetrad has been stored as e(bdn,dn).}\\

\noindent From this point we can proceed to determine the Petrov
type of our example metric by first calculating the spin coefficients
(\texttt{NPSpin} and their conjugates \texttt{NPSpinbar}) and the Weyl
scalars (\texttt{WeylSc}) \cite{newman/penrose:1962}.\\

\noindent\texttt{> grcalc ( NPSpin, NPSpinbar ):\\
> grcalc ( WeylSc ):\\
> gralter ( \_ , factor ):\\
> grdisplay ( \_ ):}
  \[ { \Psi 0}= - \,{\displaystyle \frac {5}{64}}\,{\displaystyle 
    \frac {{I}\,{x}\,(\,13\,{t}^{2} + 10\,{I}\,{t} + 11\,)}{(\,{t} + 
    {I}\,)\,(\, - {t} + {I}\,)^{2}}} \]
  \[ { \Psi 1}={\frac {-5/8\,i\sqrt {x} \left( t+i \right) {\lambda}^{2} \left( -t+i
 \right) ^{2}}{ \left( - \left( {t}^{2}+1 \right) \lambda \right) ^{5/
2}}} \]
  \[ { \Psi 2}=0 \]
  \[ { \Psi 3}=0 \]
  \[ { \Psi 4}=0 \]

\noindent Note that the calculation of \texttt{NPSpin} and \texttt{NPSpinbar}
did not need to be specified in a separate step, since this
the calculation of these objects implicit in the calculation of
\texttt{WeylSc}. However, in general a good rule of thumb is to calculate
these quantities first and simplify them fully (not required in this
case) before proceeding to the curvature scalars.\\

\noindent\texttt{> grcalc ( Petrov ):\\
> grdisplay ( \_ ):}
\begin{center}\textit{For the sm spacetime:\\
  Petrov Type = III (or simpler)}
\end{center}
\noindent Note the caveat `or simpler' in the final line of output. The
calculation of the Petrov type involves determining whether a number of
intermediate quantities can be evaluated to zero
\cite{letniowski/mclenaghan:1988}.
It is sometimes the case that complicated ways of expressing zero do not, in
fact, reduce to zero within the computer algebra system. In such cases it will 
return a Petrov type which is unnecessarily specialized. To determine the
reliability of the computer's evaluation, the \grcmd{PetrovReport} command
is provided, which details the intermediate calculations performed in
the determination of the Petrov type.\\

\noindent\texttt{> PetrovReport():}
\begin{center}\textit{
  The conclusion 'Petrov type = III (or simpler)' for the sm metric\\
  was based on the following results:\\
  Weyl scalar Psi0 could not be evaluated to zero.\\
  Weyl scalar Psi1 could not be evaluated to zero.\\
  Weyl scalar Psi2 = 0\\
  Weyl scalar Psi3 = 0\\
  Weyl scalar Psi4 = 0}\\

\textit{-- -- -- $>$ Therefore the metric is Petrov III (or simpler).
}
\end{center}

\noindent We see from the previous output that the indeed the calculations
have been carried out correctly (all of the terms which the computer believes
to be non-zero are actually non-zero).

The use of null tetrads and general bases is treated in more detail in
Booklet \grBasisRef.
%
%------------------------------------------------------------------------------
\section{Defining new tensors}
%------------------------------------------------------------------------------
Finally, we describe a facility which exists within GRTensorIII for defining
tensors which are not part of the standard object libraries. Here we
define the Bel-Robinson tensor, 
\[
  T_{cdef} := C_{acdb}C^a{}_{ef}{}^b+C^*_{acdb}C^{*a}{}_{ef}{}^b,\qquad
\]
as well as it's trace and its divergence using the command \grcmd{grdef}:\\

\noindent\texttt{> grdef (`T\{(c d e f)\} := C\{a c d b\}*C\{\^{}a e f\^{}b\}+
  Cstar\{a c d b\}\\
  \indent *Cstar\{\^{}a e f\^{}b\}`):\\
> grdef (`TT\{(c d)\} := T\{\^{}a a c d\}`):\\
> grdef (`TC\{(b c d)\} := T\{\^{}a b c d ;a\}`):}\\

\noindent In the strings defining these tensors, contravariant 
indices are indicated by preceding them with a caret, `\texttt{\^{}}',
while covariant differentiation is indicated as usual, with a
semi-colon, `\texttt{;}'.  The braces on the left-hand side of the
assignment in each command indicate that the newly defined tensors
should be assumed to be symmetric under interchange of their indices.
The calculation functions which are automatically generated by
\grcmd{grdef} will take these symmetries into account to improve the
efficiency of calculations.

We now load the Kerr-Newman metric and calculate these objects:\\

\noindent\texttt{> qload(newkn):}
\begin{center}\textit{
  Coordinates:\\
  $x^a = [\ r\ u\ \phi\ t\ ]$\\
  Line element}
\end{center}
\begin{align*}
  ds^2 = & \frac{(r^2 + u^2)dr^2}{r^2 - 2mr + a^2 + Q^2}
    + \frac{(r^2 + u^2)du^2}{a^2 - u^2}
    + \frac{(a^2 - u^2)\left(r^2 + a^2 + \frac{(a^2 - u^2)
      (2mr-Q^2)}{r^2 + u^2}\right)d\phi^2}{a^2}\\
    & -\frac{(a^2-u^2)(2mr-Q^2)d\phi dt}{a(r^2+u^2)}
    + \left( -1 + \frac{2mr - Q^2}{r^2 + u^2}\right) dt^2
\end{align*}
\begin{center}
  \textit{Kerr--Newman Solution in Boyer--Lindquist coordinates 
  ($u=a\ast \cos(\theta)$)}\\
\end{center}

\noindent\texttt{> grcalc ( T(dn,dn,dn,dn) ):}\\

\noindent We display a component (e.g. $T_{rrrr}$):\\

\noindent\texttt{> grcomponent ( T(dn,dn,dn,dn), [r,r,r,r] ):}
\[
  6\frac{m^2 r^2 - 2mrQ^2 + m^2 u^2 + Q^4}
    {(r^2 + u^2)^2(r^2 - 2mr + a^2 + Q^2)^2}
\]

\noindent We now check that $T^a{}_{acd}=0$ and that 
$\nabla_a T^a{}_{bcd}=0$ in vacuum:\\

\noindent\texttt{> grcalc ( TT(dn,dn) ):}\\
\texttt{> grdisplay( \_ ):}
\begin{center}
  \textit{TT(dn,dn) $=$ All components are zero}
\end{center}
\texttt{> Q:=0:}\\
\texttt{> grcalc ( TC(dn,dn,dn) ):}\\
\texttt{> grdisplay( \_ ):}
\begin{center}
  \textit{TC(dn,dn,dn) = All components are zero}
\end{center}

A more thorough discussion of the definition of tensors in GRTensorIII is
to be found in Booklet \grDefRef.
%
\pagebreak
%------------------------------------------------------------------------------
\section*{Conventions}
%------------------------------------------------------------------------------
GRTensorIII can perform tensor calculations in any number of dimensions
and for symmetric metrics of arbitrary signature. Tensor indices
generally take values from $1\ldots n$ in an $n$-dimensional spacetime.

In its definitions of curvature tensors, GRTensorIII follows the
Landau-Lifshitz spacelike conventions. That is, the curvature (Riemann)
tensor is defined by
\[
  R^a{}_{bcd} := \frac{\partial\Gamma_{bd}^a}{\partial x^c}
    - \frac{\partial\Gamma_{bc}^a}{\partial x^d}
    + \Gamma_{ec}^a \Gamma_{bd}^e
    - \Gamma_{ed}^a \Gamma_{bc}^e,
\]
the Ricci tensor by
\[
  R_{ab} := R^c{}_{acb},
\]
and the Einstein tensor by
\[
  G_{ab} := R_{ab} - \frac{1}{2} R g_{ab}.
\]

Although spacetimes of any signature can be defined, for four
dimensional Lorentzian metrics the signature +2 is recommended.
Most standard curvature calculations will be carried out correctly
regardless of the signature convention. However, some tensors (in
particular the vector operators of Booklet \grCalcRef) which depend on
the normalization of timelike or spacelike vectors can be affected by
the choice of signature. If a signature other than +2 is employed,
users should carefully check tensor definitions for signature
dependent terms.\\

The only deviation from this standard signature is in the specification
of the Newman-Penrose formalism, which follows the original definitions
of \cite{newman/penrose:1962}, and thus requires a -2 signature for
spacetimes.\\
%
%------------------------------------------------------------------------------
\vfill
\bibliographystyle{unsrt}
\bibliography{grtensor}
\end{document}
%==============================================================================
